\section{Introduction}

%% The importance of magnetization direction and techniques
Most of  magnetic methods require knowledge of the magnetization direction, otherwise 
they yield unsatisfactory interpretations of the exploration targets. This fact has 
been propelled the development of several techniques for estimating magnetization 
direction over the last 50 years. The strategies for estimating this quantity can be 
divided into two main groups. The first one comprises the methods which presume a prior 
information about the shape of geological sources. The iterative method presented by 
\cite{bhattacharyya1966} presumes that the magnetic source has a rectangular prismatic shape. 
\cite{emilia_massey_1974} approximate a seamount by a set of stacked prisms with uniform 
magnetization direction and variable magnetization intensity. 
\cite{parker_etal_1987} proposed a method for estimating de magnetization direction 
formulating as an optimization problem by imposing an uniformly magnetization distribution
[QUAL É A PREMISSA DESTE MÉTODO?]. 
\cite{kubota2005} also approximates a seamount by a set of juxtaposed prisms, but estimate a 
magnetization direction for each one. 
Finally, \cite{oliveirajr_etal_2015} approximate the magnetic sources by spherical bodies with 
known centers for estimating their magnetization directions. 
The second group is formed by methods which do not presume any information about the shape of the 
magnetic sources. \cite{fedi_etal_1994}, for example, proposed a method that determines the best 
magnetization direction among a set a tentative values used to perform 
successive reductions-to-the-pole on Fourier domain. 
\cite{medeiros_silva_1995} presented an interpretation method that estimates the total magnetization 
direction and the spatial orientation of the source based on the multipole expansion. [ACHO QUE ESTE MÉTODO
PERTENCE AO PRIMEIRO GRUPO PORQUE PRESUME QUE O CORPO TEM TRÊS PLANOS DE SIMETRIA MUTAMENTE ORTOGONAIS]. 
\cite{phillips2005} uses the Helbig's integral for estimating the components of the magnetic-moment vector. 
\cite{tontini_pedersen_2008} extended the Phillips' method by using the same Helbig's integral 
to estimate the magnetization direction and its magnitude, also providing information about the position 
of the center of magnetization distribution. \cite{lelievre_oldenburg_2009} developed a method for estimating 
the magnetization direction in complex geological scenarios. Their method approximates the subsurface 
by a grid of juxtaposed prisms and estimates the components of the magnetization vector for each prism.
In addition, there are methods based on the correlation of potential-field quantities \citep[e.g.,][]{dannemiller_li_2006,gerovska_etal_2009,liu_etal_2015,zhang_etal_2018}.

%%The equivalent layer 
Estimating the magnetization direction is extremely important not only for interpretation, 
but also for processing the total-field anomaly data. One technique in spatial domain 
commonly used for processing potential-field data is the equivalent layer. It was first introduced 
in exploration geophysics by \cite{dampney1969} and \cite{emilia_massey_1974} for processing 
gravity and magnetic data, respectively. After these pioneer works, this technique has been widely 
used for computing interpolation \citep{cordell_1992, mendonca-silva_1994, barnes-lumley_2011, siqueira_etal_2017}, 
upward (or downward) continuation  \citep{hansen-miyazaki_1984, li-oldenburg_2010}, reduction to the pole 
\citep{silva_1986, leao-silva_1989, guspi-novara_2009, oliveirajr-etal_2013}, the amplitude of 
anomalous field \citep{li_li_2014} and for denoising gradient data \citep{martinez_li_2016}. 
The equivalent-layer technique consists in approximating the observed data by that produced by a 
layer of discrete sources (e.g., prisms, dipoles or point masses), which are commonly known as 
equivalent sources. The data produced by this fictitious layer (the equivalent layer) are commonly
called predicted data.

%%  positivity constraints 
In scanning magnetic microscopy, the equivalent-layer technique is generally used for interpreting 
the magnetic-moment distribution whithin thin planar sections of rock samples. Notice that, in this case, 
the equivalent layer resembles the true source (a thin section of rock). 
\cite{weiss2007} presented one of the first works using the equivalent-layer technique in scanning magnetic microscopy.
They pointed out without proof that the estimated magnetic-moment distribution on the layer is all-positive if 
the magnetization direction of the equivalent sources is equal to the one used for artificially magnetizing the rock sample. 
\cite{baratchart2013} show mathematically that, assuming a uniform magnetization direction within the thin section, 
the inverse problem of estimating the magnetic-moment distribution has uniqueness. 
\cite{lima2013} proposed a method on the frequency domain to investigate solutions having a uniform magnetization 
direction equal to that of a thin section of geological sample. They show empirically 
that, in this case, the estimated magnetic-moment distribution on the layer is entirely positive. 

In the geophysical exploration, the equivalent-layer technique is predominantly used for processing potential-field data. 
Under this perspective, there is no relationship between the physical-property distribution on the equivalent layer 
and the true geologic sources. Hence, the layer is just a mathematical abstraction devoid of geological meaning. 
Few authors in geophysical exploration literature have been addressed the use of the equivalent-layer technique for interpreting
the geologic sources. \cite{pedersen1991}, for example, discussed the relationship between potential field and equivalent source. 
\cite{medeiros_silva1996} and \cite{silvadias_etal_2010} estimated an apparent-magnetization map on a layer by using Tikhonov and 
entropic regularizations respectively. \cite{siqueira_etal_2017} established a relationship between the excess of mass estimated 
over the equivalent layer and the true one. \cite{li_nabighian_oldenburg_2014} proved, by using an approach in the Fourier domain, 
the existence of an all-positive magnetic-moment distribution over the layer and use this to overcome the low-latitude instability. 
However, these authors considered only the particular case in which the magnetic sources have a purely induced magnetization.

%% Joining techniques for estimating the magnetization direction 
Here, we prove mathematically that the all-positive magnetic-moment distribution within an equivalent layer exists even in the 
presence of remanent magnetization of the true geologic sources. This all-positive magnetic-moment distribution holds true for 
all cases in which the magnetization direction of the equivalent sources has the same direction as that of the true geologic sources, 
regardless of whether the magnetization of the true sources is purely induced or not. Grounded on this generalized positivity constraint, 
we present a new iterative method that uses the equivalent-layer technique for estimating the uniform magnetization direction of 
arbitrary sources by inverting total-field anomaly data. Our method does not presume any information about the shape of the sources. 
At each iteration, our method solves (i) a linear inverse problem, subjected to a positivity constraint, for estimating the magnetic-moment 
distribution within a planar equivalent layer of dipoles, and (ii) a non-linear inverse problem for estimating the uniform magnetization 
direction of the equivalent sources. Tests with synthetic data generated by different geological scenarios show that the estimated magnetization
direction converges to that of the true sources. We also applied our method to field data from the Goi{\' a}s alkaline province (GAP), 
over the Montes Claros complex, center of Brazil. Our result is in agreement with that obtained independently by \cite{zhang_etal_2018} 
at the same area, suggests the presence of a remarkable remanent magnetization and shows the good performance of our method in interpreting 
a complex geological scenario.
