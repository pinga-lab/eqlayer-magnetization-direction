\section{Introduction}

%% Brief introduction

The magnetic method is one of the oldest geophysical technique that plays an important role in exploration geophysics. This method has been applied for interpreting a vast geological scenarios like estimating the basement relief, mapping faults and lithological contacts, defining bounds of geological sources, determining the position of salt domes whithin sediments, and indentifying oil and gas traps. With the improvement of magnetic measurements, mainly by the use of aeromagnetic surveys providing a large amount of data, this technique has become standard in geophysics to map areas with a variety of scales \citep{blakely1996,nabighian_etal_2005}. In order to better describe the subsurface structures, it is necessary to extract a reliable information from magnetic data. 

%% The importance of magnetization direction and techniques
The magnetization direction is very important to interpret magnetic anomalies allowing to characterize the exploration targets. Hence, several techniques for estimating magnetization direction have been emerged through the last years. The strategies for estimating this quantity can be divided into two main groups. The first one comprises the methods which presume a priori information about the shape of geological sources. \cite{bhattacharyya1966} presented an iterative method for determining the parameters of a rectangular prismatic shape, such as horizontal dimensions, depth of the body and magnetization direction. \cite{emilia_massey_1974} approximate the a seamount by a set of stacked prism with the same magnetization direction and different magnetization intensity. \cite{parker_etal_1987} proposed a method for estimating de magnetization direction formulating as an optimization problem by imposing an uniformly magnetization distribution. \cite{kubota2005} also parametrized a seamoumt approximating by a set of juxtaposed prisms and estimating the magnetization direction of each one. Finally, \cite{oliveirajr_etal_2015} approximate a magnetic source by a spherical geometry, assumig the knowledge of the center and estimating the magnetization direction. The second group are the methods which not presume any information about the magnetic source. \cite{fedi_etal_1994}, for example, proposed a method that performs a succesive reduction-to-pole on frequency domain testing it for a set of magnetization direction. \cite{medeiros_silva_1995} presented an interpretation method that estimates the total magnetization direction and the spatial orientation of the source based on the multipole expansion. \cite{phillips2005} implemented an algorithm that uses the Helbig's integral for estimating the three-component vector of the magnetic-dipole moment. \cite{tontini_pedersen_2008} extended the method using the same Helbig's integral to estimate the magnetization direction and its magnitude, also providing information about the position of the centre of magnetization distribution. \cite{lelievre_oldenburg_2009} developed a method for estimating the magnetization direction in complex geological scenarios. In addition, there are the methods that are based on the correlation of potential quantities \citep{dannemiller_li_2006,gerovska_etal_2009,liu_etal_2015,zhang_etal_2018}.

%%The equivalent layer and  positivity constraints 
Estimating the magnetization direction is extremely important not only for interpretation, but also for processing the total-field anomaly data. One technique in spatial domain commonly used for processing potential-field data is the equivalent layer. It was first introduced by \cite{dampney1969} and \cite{emilia_massey_1974} for processing gravity and magnetic data, respectively. After these pioneer works, this technique has been widely used for processing such as interpolation \citep{cordell_1992,mendonca-silva_1994,barnes-lumley_2011,siqueira_etal_2017}, upward (or downward) continuation  \citep{hansen-miyazaki_1984,li-oldenburg_2010}, reduction to the pole \citep{silva_1986,leao-silva_1989,guspi-novara_2009,oliveirajr-etal_2013}, computing the amplitude of anomalous field \citep{li_li_2014} and denoising gradient data \citep{martinez_li_2016}. The equivalent layer approach consists to approximate the observed and predicted data produced by a set of discrete sources (e.g., prisms, dipoles or point masses) commonly known as equivalent sources. Once the physical property in the layer is estimated, it can be used for processing potential-field data. Nevertheless, prominent discussions about the physical-property estimated over the layer are increasing in the last decades. For example, \cite{pedersen1991} discussed the relation between potential field and equivalent source. \cite{weiss2007}, using magnetic microscopy data, pointed out that fixing a magnetization direction the magnetic-moment estimated in the layer are all-positive. \cite{baratchart2013} show that, assuming the unidirectional solution, its possible to achieve uniqueness on the inverse problem. \cite{lima2013} proposed a method on the frequency domain to investigate the unidirectional solution on a planar geological sample with magnetic microscopy data, showing that the physical-property estimated over the layer is entirely positive. On the other hand, \cite{li_nabighian_oldenburg_2014} using total-field anomaly data, proved the existence of a positive distribution over the layer, and its a sufficient feature to overcome the low-latitude instability. However, these authors considered only a purely induced magnetization for the magnetic sources.

%% Joining techniques for estimating the magnetization direction 
We present a new method using total-field anomaly data that estimates the magnetization direction without presume any information about the magnetic sources. It is based on the exploration of the theoretical aspect of positivity to build a nested algorithm to solve the inverse problem. We examine the performance of the method testing it in synthetic tests generated by complex geological scenarios. Furthermore, application to field data from Goiás alkaline province (GAP) over the Montes Claros complex, center of Brazil, shows the performance of the method in estimating a meaningful magnetization direction. The result suggests the presence of a remarkable remanent magnetization.    

 


%Neste sentido, propomos um novo método para estimar a direção de magnetização de fontes magnéticas utilizando dados de anomalia de campo total. Exploramos o aspecto teórico da positividade para elaborarmos um esquema iterativo para resolver este problema inverso. Aplicações a dados sintéticos mostram que o método foi capaz de recuperar a magnetização total das fontes. Além disso, aplicamos a metodologia para dados de campo provenientes da Província Alcalina de Goiás (PAGO), na região central do Brasil.  Os resultados desta aplicação mostraram que o método é capaz de estimar uma direção de magnetização geologicamente coerente com a área de estudo, sugerindo a existência de fontes com forte presença de magnetização remanente, mostrando estar de acordo com alguns trabalhos anteriores. 