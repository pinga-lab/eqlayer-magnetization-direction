\section{Conclusion and discussion}
\label{sec:conclusion}

We have presented a new method for estimating the total magnetization direction of 3D magnetic sources by using the equivalent-layer technique. Our method is formulated as an iterative least-squares problem and imposes a positivity constraint on the magnetic moments over the layer. Prior knowledge about the shape and depth of magnetic sources are not required, neither the use of an envenly spaced data. This methodology can be applied for determining the magnetization direction within multiple sources, considering all of them have the same magnetization direction. Moreover, we show the theoretical proof of the positive characteristic of the equivalent layer when it has the same magnetization direction of the true magnetic source, even if its purely induced or not.

By imposing the positivity constraint for magnetic-moment distribution allows the equivalent-layer to provide informations about the magnetization direction of magnetic sources. The results obtained with the synthetic data produced by unidirectional model have shown the good performance of our method for retrieving the true magnetization direction. Application to field data over the Goias alkaline province (GAP), center of Brazil, has confirmed that our method can be a powerful tool for interpreting real complex geological scenarios as well. The application for the complex of Montes Claros suggests the presence of a strong remanent magnetization component, in accordance to previous studies for the same magnetic anomaly. Moreover, the all-positive magnetic moment over the layer leads to very plausible RTP anomalies. However, both two synthetic tests and the real data application present a marked residuals in some locations over the map of difference between the observed and predicted data. We consider that this markable feature was caused by shallow interfering sources. After some synthetic tests, regardless the shallow body has the same direction of the other sources, it can be produce a data misfit just above the region of an interfering body. Despite reliable results for magnetization direction estimation, we cannot infer if the shallow source has the same direction of other bodies or not. For this reason, it is necessary a further analysis for interpreting the anomalies caused by shallow interferences.  
