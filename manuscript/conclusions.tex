\section{Conclusion}
\label{sec:conclusion}

We have shown mathematically that the total-field anomaly data caused by a set of magnetic sources
with uniform magnetization direction can be exactly reproduced by a continuous and planar layer of 
dipoles having an all-positive magnetic-moment distribution. 
This theoretical property holds true for the case in which the layer has the same
magnetization direction as that of the true sources, regardless of whether they have 
a purely induced magnetization or not.
By using this generalized positivity constraint, we have presented a new iterative method for 
estimating the total magnetization direction of 3D magnetic sources based on the equivalent-layer technique. 
At each iteration, we impose a positivity constraint on the estimated magnetic-moment distribution of the layer 
and solves a non-linear inverse problem for estimating the magnetization direction of the equivalent sources. 
Prior knowledge about the shape and depth of magnetic sources are not required, neither the use of an evenly spaced 
data set. This methodology can be applied for determining the magnetization direction of multiple sources, 
considering all of them with the same magnetization direction. 
Results obtained with synthetic data produced by multiple sources have shown that the estimated magnetization direction 
obtained by our iterative method successfully retrieves the true one.
Tests with synthetic data have also illustrated how the presence of a relatively shallow-seated source affects the 
result obtained by our method for the cases in which it has a magnetization direction equal to and different from 
the other sources. In both cases, the equivalent layer yielded large data misfits above the shallow source;
however, we cannot distinguish if the shallow source has a magnetization direction equal or different from the other sources. 
Moreover, our method produces the worst estimated magnetization direction when shallow-seated source is magnetized in a 
direction that differs from the other sources.
An application to field data over the Goi{\' a}s alkaline province, center of Brazil, has confirmed that our method can 
be a reliable tool for interpreting complex geological scenarios. The result over the Montes Claros complex suggests the presence 
of a strong remanent magnetization component and corroborates a previous study conducted independently at the same area. 
The estimated magnetic-moment distribution over the layer has led to a very acceptable reduction to the pole, but have also 
produced large data-misfits at some isolated regions. We presume that these locally large data-misfits are due to shallow sources, 
however we cannot infer if they have the same magnetization direction of the other bodies.   
