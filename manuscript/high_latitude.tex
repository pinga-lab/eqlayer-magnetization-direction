\append{Vertically magnetized sources}
\label{append:vertical-magnetization}

A limitation of our method is its slow convergence for the case in which the sources have
a vertical magnetization. In this appendix, we provide the theoretical basis for understanding 
this problem.

Consider the $N \times 2$ matrix $\mathbf{G}_{q}^{k}$ (equation \ref{eq:Gq}) forming the 
nonlinear system required for estimating the correction $\bar{\mathbf{\Delta q}}^{k}$ 
on the magnetization direction (equation \ref{eq:linear_sys_q}).

Its elements are defined by the first derivatives 
$\partial_{\alpha} \mathbf{g}_{i}(\bar{\mathbf{q}}^{k}) \equiv 
\frac{\partial \mathbf{g}_{i}(\bar{\mathbf{q}}^{k})}{\partial \alpha}$, $\alpha= I, D$, 
of the vector $\mathbf{g}_{i}(\mathbf{q})$ (equation \ref{eq:tfa_pred_i}), 
evaluated at $\mathbf{q} = \bar{\mathbf{q}}^{k}$, with respect to the 
inclination $I$ and the declination $D$ of the total magnetization of the sources.

From equation \ref{eq:g_ij}, these derivatives are given by:
\begin{equation}
\partial_{\alpha} \mathbf{g}_{i}(\bar{\mathbf{q}}^{k}) = 
\gamma_m  \begin{bmatrix}
\hat{\mathbf{F}}_{0}^T \, \mathbf{M}_{i1} \\
\vdots \\
\hat{\mathbf{F}}_{0}^T \, \mathbf{M}_{iM}
\end{bmatrix}
\partial_{\alpha} \hat{\mathbf{m}}(\bar{\mathbf{q}}^{k}) \: , \quad \alpha = I, D \: ,
\label{eq:D-alpha-gi}
\end{equation}
where $\hat{\mathbf{F}}_{0}$ (equation \ref{eq:main_field}) is the unit vector
defining the direction of the main field, $\mathbf{M}_{ij}$, $j = 1, \dots, M$,
is the $3 \times 3$ matrix defined by equation \ref{eq:Mij-matrix} and 
$\partial_{\alpha} \hat{\mathbf{m}}(\bar{\mathbf{q}}^{k}) \equiv 
\frac{\partial \hat{\mathbf{m}}(\bar{\mathbf{q}}^{k})}{\partial \alpha}$, $\alpha= I, D$, 
is the first derivative of the unit vector $\hat{\mathbf{m}}(\mathbf{q})$ (equation \ref{eq:q_vector}),
evaluated at $\mathbf{q} = \bar{\mathbf{q}}^{k}$, with respect to the inclination $I$ or declination $D$.

\begin{equation}
\partial_{I} \hat{\mathbf{m}}(\bar{\mathbf{q}}^{k}) = 
\begin{bmatrix}
	-\sin I \cos D \\
	-\sin I \sin D\\
	 \cos I
\end{bmatrix}
\label{eq:D_mag_vec_inc}
\end{equation}

\begin{equation}
\partial_{D} \hat{\mathbf{m}}(\bar{\mathbf{q}}^{k}) = 
\begin{bmatrix}
	-\cos I \sin D \\
	 \cos I \cos D\\
	 0
\end{bmatrix}
\label{eq:D_mag_vec_dec}
\end{equation}


Hence, for the case in which the total-magnetization of the sources has inclination $I = 90^\circ$, we have 
\begin{equation*}
\partial_{I} \hat{\mathbf{m}}(\bar{\mathbf{q}}^{k}) = 
\begin{bmatrix}
	-\cos D \\
	-\sin D\\
	 0
\end{bmatrix}
\label{eq:D_mag_vec_inc_I90}
\end{equation*}
and
\begin{equation*}
\partial_{D} \hat{\mathbf{m}}(\bar{\mathbf{q}}^{k}) = 
\begin{bmatrix}
	0 \\
	0 \\
	0
\end{bmatrix} \: .
\label{eq:D_mag_vec_dec_I90}
\end{equation*}

We can notice in this case that the second column of the matrix $\mathbf{G}_{q}^{k}$ (equation \ref{eq:Gq}) 
has all elements equal to zero, leading to an ill-posed inverse problem.         



