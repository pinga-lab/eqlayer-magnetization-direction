\append{Consequences of high-latitude estimation}

One critical limitation for estimating the magnetization direction results of the case of estimating magnetization direction in high latitudes. In this appendix, we prove the existence of high latitude difficulties providing a theoretical basis and testing it on extreme cases.

The process of estimating magnetization direction using equivalent-layer technique is divided into two inversions. One by solving a constrained linear problem using positivity for magnetic-moment distribution and the other an unconstrained nonlinear estimation for magnetization direction. A way to extract infomations about the rank-deficiency and ill-conditioning of linear systems is exploring the singular value decomposition. This procedure allows an $N \times M$ matrix can be rewritten as the product of three matrices, one of them formed by the set of singular values arranged in order of decreasing size. The context of decomposing a matrix in singular values is to obtain pieces of information that can be estimated. It can also be useful to dictate which rows or columns of the matrix are linear independent. The fact is that, if a row or a column of a matrix is all null, it impacts directly on the linear dependence. Consequently, it leads to a rank-deficiency and an ill-conditioning of a linear system. Thus, the construction of the $N \times 2$ sensitivity matrix $\mathbf{G}_{p}^{k}$ is explicitly given by

\begin{equation}
\mathbf{G}_{p}^{k} =
\left[ \begin{array}{cc}
\mathbf{p}^{k^T} \partial_{\tilde{\i}} \mathbf{g}_{1} (\mathbf{q}^k) & \mathbf{p}^{k^T} \partial_{\tilde{d}} \mathbf{g}_{1}(\mathbf{q}^k) \\
\vdots & \vdots  \\
\mathbf{p}^{k^T} \partial_{\tilde{\i}} \mathbf{g}_{N} (\mathbf{q}^k) & \mathbf{p}^{k^T} \partial_{\tilde{d}} \mathbf{g}_{N} (\mathbf{q}^k)
\end{array} \right] ,
\label{eq:gp}
\end{equation}
in which $\mathbf{p}^{k^T} \partial_{\alpha} \mathbf{g}_{i} (\mathbf{q}^k)$, $\alpha =\tilde{\i}, \tilde{d}$, is a derivative in relation to inclination (first column) and the declination (second column) evaluated at the $i$th observation point. That is, by calculating the derivative of equation \ref{eq:f_i} in relation to magnetization direction is in fact the derivative of the vector $\hat{\mathbf{m}}(\mathbf{q})$ (equation \ref{eq:mag_vec}). That is, the derivative of the vector $\hat{\mathbf{m}}(\mathbf{q})$ in relation to inclination is given by

\begin{equation}
	\dfrac{\partial \hat{\mathbf{m}}(\mathbf{q})}{\partial {\tilde{\i}}} =
	\left[ \begin{array}{c}
		-\sin \tilde{\i} \cos \tilde{d} \\
		-\sin \tilde{\i} \sin \tilde{d}\\
		 \cos \tilde{\i}
	\end{array} \right], 
	\label{eq:mag_vec_inc}
\end{equation}
and analogously for the declination is equal to 

\begin{equation}
	\dfrac{\partial \hat{\mathbf{m}}(\mathbf{q})}{\partial {\tilde{d}}} =
	\left[ \begin{array}{c}
		-\cos \tilde{\i} \sin \tilde{d} \\
		\cos \tilde{\i} \cos \tilde{d}\\
		0
	\end{array} \right]. 
	\label{eq:mag_vec_dec}
\end{equation}

Hence, for the case whose the magnetization direction of the true magnetic source has inclination $90^\circ$ and declination $0^\circ$, the equation \ref{eq:mag_vec_inc} and \ref{eq:mag_vec_dec} is, respectively, equal to 

 
\begin{equation}
	\dfrac{\partial \hat{\mathbf{m}}(\mathbf{q})}{\partial {\tilde{\i}}} =
	\left[ \begin{array}{c}
		-1 \\
		 0 \\
		 0
	\end{array} \right], 
	\label{eq:mag_vec_inc_90}
\end{equation}
and 

\begin{equation}
	\dfrac{\partial \hat{\mathbf{m}}(\mathbf{q})}{\partial {\tilde{d}}} =
	\left[ \begin{array}{c}
		0 \\
		0 \\
		0
	\end{array} \right]. 
	\label{eq:mag_vec_dec_0}
\end{equation}
We can notice from all these reasoning that the second row of the sensitivity matrix \ref{eq:gp} is all null. Consequently, it leads to a rank-deficiency and the ill-conditioning of a linear system when the true magnetization direction is vertical. This situation is different for the extreme case of low latitude, in which the magnetization direction has the inclination $0^\circ$ and declination $0^\circ$. 

           



