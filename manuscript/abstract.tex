\begin{abstract}
We developed a new method for estimating the total magnetization direction of magnetic sources based on equivalent layer technique using total-field anomaly data. In this approach, we do not have to impose strong information about the shape and the depth of the sources neither require a regularly spaced data. Usually, this technique is used for processing potential data estimating a 2D magnetic-moment distribution over a fictitious layer composed by dipoles below the observation plane. In certain conditions, when the magnetization direction of equivalent sources is almost the same as the true body, the estimated magnetic property over the layer is all positive. Our method uses this remarkable feature to estimate a nonnegative magnetic-moment distribution over the layer and a magnetization direction through an iterative process. Therefore, we propose a nested algorithm to solve the inverse problem in two steps. We test the methodology by applying to synthetic data for complicated geological scenarios. Moreover, we applied the method to field data from Goias Alkaline Province (GAP), center of Brazil, over Montes Claros complex. 
\end{abstract}