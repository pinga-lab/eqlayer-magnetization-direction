\begin{abstract}
We have developed a new methor for estimating the total magnetization direction of magnetic sources based on equivalent layer technique using total field anomaly data. In this approach, we do not have to impose a strong information about the shape and the depth of the sources, and do not require a regularly spaced data. Usually, this technique is used for processing potential data estimating a 2D magnetic moment distribution over a ficticious layer composed by dipoles below the observation plane. In certain conditions, when the magnetization direction of equivalent sources is almost the same of true body, the estimated magnetic property over the layer is all positve. The methodology uses a positivity constraint to estimate a set of magnetic moment and a magnetization direction of the layer through a iterative process. Mathematically, the algorithm solve a least squares problem in two steps: the first one solve a linear problem for estimating a magnetic moment and the second solve a non-linear problem for magnetization direction of the layer. We test the methodology applying to synthetic data for different geometries and magnetization types of sources. Moreover, we applied this method to field data from Goias Alkaline Province (GAP), center of Brazil, showing that the methodology can be a good tool for estimating the magnetization component of the alkaline intrusion complex in  the Diorama region. The result for this complex suggests that this source has a remarkable strong remanent magnetization component. The magnetization direction estimated for this complex is  $-47^\circ$ and $-111^\circ$ for inclination and declination, respectivelly. 


\end{abstract}