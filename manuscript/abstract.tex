\begin{abstract}
We developed a new method for estimating the total magnetization direction of magnetic sources based on equivalent-layer technique using total-field anomaly data. This approach does not impose strong information either about the shape or about the depth of the sources, and does not require a regularly spaced data. Usually, this equivalent-layer technique is used for processing total-field anomaly data by estimating a 2D magnetic-moment distribution over a fictitional layer composed by dipoles below the observation plane. When the magnetization direction of equivalent sources is almost the same as the true body, the estimated magnetic property over the layer is all positive. Iteratively, the proposed method imposes zeroth-order Tikhonov regularization and positivity constraint on the estimated magnetic moment over the layer and estimate the magnetization direction of the geological sources. Mathematically, the algorithm solves least-squares problems in two steps: the first one solves a linear inverse problem for estimating a 2D magnetic-moment distribution within the equivalent layer and the second solves a nonlinear inverse problem for magnetization direction of the magnetized sources. We test the methodology by applying to synthetic data for complicated geological scenarios, and the results show that the method can be a powerful tool for estimating the magnetization direction of a set of bodies. Tests on field data from Goias Alkaline Province (GAP), center of Brazil, over Montes Claros complex suggests intrusions with remarkable strong remanent magnetization, in agreement with the current literature for this region. 
\end{abstract}