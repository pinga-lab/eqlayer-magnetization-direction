\section{Methodology}
\label{sec:methodology}

\subsection{Fundamentals of magnetic equivalent layer}
\label{subsec:mag_eqlayer}

% % Explaining the continuous magnetic equivalent layer  

Considering a Cartesian coordinate system with $x$-, $y$- and $z$-axis being oriented to north, east and downward, respectively. Let $\Delta T_i \equiv \Delta T (x_i,y_i,z_i)$ be the total field anomaly, at the $i$-th position $(x_i,y_i,z_i)$, produced by a continuos layer located below the observation plane on the depth $z_c$, where $z_c > z_i$, and $p(x',y',z_c)$ is the distribution of magnetic dipoles per unit area over the layer surface. In this case, the total field anomaly produced by this continuous layer is given by the equation 

\begin{equation}
\Delta T_i = \int \limits_{-\infty}^{+\infty } \int \limits_{-\infty}^{+\infty }  p(x',y',z_c)  [\gamma_m \hat{\mathbf{F}}_0^T \mathbf{H} \,\hat{\mathbf{h}}(\mathbf{q})] dx' \,dy',
\label{eq:continuous_layer}
\end{equation}
where $\gamma_m$ is a constant proportional to the vaccum permeability, $\hat{\mathbf{F}}_0$ is a unit vector with the same direction of the geomagnetic field $\mathbf{F}_0$ and $\mathbf{H}$ is a $3 \times 3$ matrix equal to  

 \begin{equation}
   \mathbf{H} =
   \left[ \begin{array}{ccc}
   \partial_{xx} \phi & \partial_{xy} \phi &\partial_{xz} \phi \\  \partial_{yx} \phi & \partial_{yy} \phi &\partial_{yz} \phi \\  \partial_{zx} \phi &\partial_{zy}\phi  & \partial_{zz} \phi    
   \end{array} \right] ,
   \label{eq:H}
 \end{equation}
where $\partial_{\alpha \beta}\phi$, $\alpha = x, y, z$, $\beta = x, y, z$, is the second derivative of the function 

\begin{equation}
   \phi (x-x', y-y', z-z_c) = \frac{1}{r} ,
   \label{eq:phi}
 \end{equation}
where $r = [(x-x')^2 + (y-y')^2 + (z-z_c)^2]^{1/2}$ and $\hat{\mathbf{h}}(\mathbf{q})$ is a unit vector with the
magnetization direction of the layer that depends on the vector $\mathbf{q}$ given by 

 \begin{equation}
   \mathbf{q} =
   \left[ \begin{array}{c}
   i  \\ 
   d     
   \end{array} \right] ,
   \label{eq:q_vector}
 \end{equation}
 where $i$ and $d$ is the inclination and declination, respectively.

% % Formulating foward problem 

According the theory, we can reproduce a set of $N$ observed total field anomaly produced by a 3D magnetic source using a bidimensional physical-property distribution. In practical situations, the equivalent layer is composed by a set of $M$ equivalent sources distributed with a constant depth $h$ below the observation plane. It is worth pointing out that, in this work, the equivalent source is represented by a dipole with unit volume. For this reason, the vector $\mathbf{p}$ is the \textit{M}-dimensional vector defined as parameter vector, whose $j$th element is the magnetic intensity of the $j$th equivalent source, and the vector $\mathbf{q}$ contains the inclination and the declination of each equivalent dipole. By discretizing the integrand of equation \ref{eq:continuous_layer} in a set of points $(x_j,y_j,z_c)$, $j = 1, \ldots, M$, the integral can be given by

\begin{equation}
\Delta T_i (\mathbf{p},\mathbf{q})   = \sum_{j=1}^{M} p_j g_{ij} (\mathbf{q})
\label{eq:tfa_pred_pos_i}
\end{equation}    
where $p_j$ is the magnetic moment of $j$th equivalent source and 

\begin{equation}
g_{ij} (\mathbf{q})  = \gamma_m \hat{\mathbf{F}}_0^T \mathbf{H}_{ij} \hat{\mathbf{h}}(\mathbf{q})
\label{eq:g_ij}
\end{equation}
is a harmonic function that depends on the direction $\mathbf{q}$ of the dipole and the matrix $\mathbf{H}_{ij}$ is formed by the second derivatives of a function $\phi_{ij}$ that depends on $r_{ij} = [(x_i-x_j)^2 + (y_i-y_j)^2 + (z_i-z_c)^2]^{1/2}$, analogously to equation \ref{eq:H} and \ref{eq:phi}.

Equation \ref{eq:tfa_pred_pos_i} represents the equivalent layer appproach. It is represented by the sum of the total field anomaly at the observation point $(x_i,y_i,z_i)$ produced by a set of $M$ ficticious equivalent sources, that is in this case a set of dipoles of unit volume, distributed on a horizontal plane at a constant depth $z_c$, each one with magnetic moment $p_j$ and magnetization direction $\mathbf{q}$. In matrix notation, the equation \ref{eq:tfa_pred_pos_i} can be represented as 

\begin{equation}
\Delta T = \mathbf{G}(\mathbf{q}) \mathbf{p}
\label{eq:tfa_pred}
\end{equation}
where $\mathbb{G}$ is $N \times M$ matrix composed by the elements $g_{ij}$ of the equation \ref{eq:g_ij}.

% % Formulating the inverse problem explaining the iterative method

%Let $\mathbf{\Delta T}^o$ be an \textit{N}-dimensional vector whose $i$th element is the total field anomaly observation produced by a magnetic source at the point $(x_i,y_i,z_i)$, $i = 1, \ldots, N$. 

% Constraining the magnetic moment of equivalent sources

% % Practical Procedures 

% % Generalization of all-positive equivalent sources 




