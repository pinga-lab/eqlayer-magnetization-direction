\section{Methodology}
\label{sec:methodology}

\subsection{Fundamentals of magnetic equivalent layer}
\label{subsec:mag_eqlayer}

Considering a Cartesian coordinate system with $x$-, $y$- and $z$-axis being oriented to north, east and downward, respectively. Let $\Delta T_i \equiv \Delta T (x_i,y_i,z_i)$ be the total field anomaly, at the $i$th position $(x_i,y_i,z_i)$, produced by a continuos layer located below the observation plane on the depth $z_c$, where $z_c > z_i$, and $p(x',y',z_c)$ is the distribution of magnetic dipoles per unit area over the layer surface. In this case, the total field anomaly produced by this continuous layer is given by the equation 

\begin{equation}
\Delta T_i = \int \limits_{-\infty}^{+\infty } \int \limits_{-\infty}^{+\infty }  p(x',y',z_c)  [\gamma_m \hat{\mathbf{F}}_0^T \mathbf{H} \,\hat{\mathbf{h}}(\mathbf{q})] dx' \,dy',
\label{eq:continuous_layer}
\end{equation}
where $\gamma_m$ is a constant proportional to the free space permeability, $\hat{\mathbf{F}}_0$ is a unit vector with the same direction of the geomagnetic field given by

\begin{equation}
	\hat{\mathbf{F}}_0 =
	\left[ \begin{array}{c}
		 \cos I \cos D \\
		 \cos I \sin D \\
		 \sin I     
	\end{array} \right] ,
	\label{eq:main_field}
\end{equation}
where $I$ and $D$ are the inclination and declination of main field, respectively. The $\mathbf{H}$ is a $3 \times 3$ dimensional matrix equal to  

 \begin{equation}
   \mathbf{H} =
   \left[ \begin{array}{ccc}
   \partial_{xx} \phi & \partial_{xy} \phi &\partial_{xz} \phi \\  \partial_{yx} \phi & \partial_{yy} \phi &\partial_{yz} \phi \\  \partial_{zx} \phi &\partial_{zy}\phi  & \partial_{zz} \phi    
   \end{array} \right] ,
   \label{eq:H}
 \end{equation}
where $\partial_{\alpha \beta}\phi$, $\alpha = x, y, z$, $\beta = x, y, z$, is the second derivative of the function 

\begin{equation}
   \phi (x-x', y-y', z-z_c) = \frac{1}{[(x-x')^2 + (y-y')^2 + (z-z_c)^2]^{\frac{1}{2}}} .
   \label{eq:phi}
 \end{equation}
The $\hat{\mathbf{h}}(\mathbf{q})$ is a unit vector with the magnetization direction of the layer given by 

\begin{equation}
	\hat{\mathbf{h}}(\mathbf{q}) =
	\left[ \begin{array}{c}
		\cos \tilde{\i} \cos \tilde{d} \\
		\cos \tilde{\i} \sin \tilde{d}\\
		\sin \tilde{\i}
	\end{array} \right] 
	\label{eq:mag_vec}
\end{equation}
and $\mathbf{q}$ is a vector with components given by 

 \begin{equation}
   \mathbf{q} =
   \left[ \begin{array}{c}
   \tilde{\i} \\ 
   \tilde{d} 
   \end{array} \right] ,
   \label{eq:q_vector}
 \end{equation}
where $\tilde{\i} $ and $\tilde{d} $ is the inclination and declination of magnetization of the layer, respectively.

According to the theory, we can reproduce a set of $N$ observed total field anomaly produced by a 3D magnetic source using a bidimensional physical-property distribution. In practical situations, the equivalent layer is composed by a set of $M$ equivalent sources distributed with a constant depth $h$ below the observation plane. It is worth pointing out that, in this work, the equivalent source is represented by a dipole with unit volume. For this reason, the vector $\mathbf{p}$ is the \textit{M}-dimensional vector defined as parameter vector, whose $j$th element is the magnetic intensity of the $j$th equivalent source, and the vector $\mathbf{q}$ contains the inclination and the declination of each equivalent dipole. By discretizing the integrand of equation \ref{eq:continuous_layer} in a set of points $(x_j,y_j,z_c)$, $j = 1, \ldots, M$, the integral can be given by

\begin{equation}
\Delta T_i (\mathbf{p},\mathbf{q})   = \sum_{j=1}^{M} p_j g_{ij} (\mathbf{q})
\label{eq:tfa_pred_pos_i}
\end{equation}    
where $p_j$ is the magnetic moment of $j$th equivalent source and 

\begin{equation}
g_{ij} (\mathbf{q})  = \gamma_m \hat{\mathbf{F}}_0^T \mathbf{H}_{ij} \hat{\mathbf{h}}(\mathbf{q})
\label{eq:g_ij}
\end{equation}
is a harmonic function that depends on the direction $\mathbf{q}$ of the dipole and the matrix $\mathbf{H}_{ij}$ is formed by the second derivatives of a function $\phi_{ij}$ that depends on the inverse of the function $r_{ij} = [(x_i-x_j)^2 + (y_i-y_j)^2 + (z_i-z_c)^2]^{1/2}$, analogously to equation \ref{eq:H} and \ref{eq:phi}.

Equation \ref{eq:tfa_pred_pos_i} represents the equivalent layer appproach. It is represented by the sum of the total field anomaly at the observation point $(x_i,y_i,z_i)$ produced by a set of $M$ ficticious equivalent sources, that is in this case a set of dipoles of unit volume, distributed on a horizontal plane at a constant depth $z_c$, each one with magnetic moment $p_j$ and magnetization direction $\mathbf{q}$. In matrix notation, the equation \ref{eq:tfa_pred_pos_i} can be represented as 

\begin{equation}
 \mathbf{\Delta T} (\mathbf{p}, \mathbf{q}) = \mathbf{G}(\mathbf{q}) \mathbf{p}
\label{eq:tfa_pred}
\end{equation}
where $\mathbf{G}$ is $N \times M$ matrix composed by the elements $g_{ij}$ of the equation \ref{eq:g_ij}.

% % Formulating the inverse problem explaining the iterative method

\subsection{Iterative process for magnetization estimation}

Let $\mathbf{\Delta T}^o$ be an \textit{N}-dimensional vector whose $i$th element $\Delta T_i^o$ is the total field anomaly observation produced by a magnetic source at the point $(x_i,y_i,z_i)$, $i = 1, \ldots, N$. The estimation of a set of magnetic moments $\mathbf{p}$ and the magnetization direction $\mathbf{q}$ consists to solve a inverse problem of minimizing the difference between the observed total field anomaly $\mathbf{\Delta T}^o$ and the predicted data $\Delta T (\mathbf{p}, \mathbf{q})$ by the equivalent layer (equation \ref{eq:tfa_pred}). In other words, a stable estimates $\mathbf{p}^\sharp$ and $\mathbf{q}^\sharp$ can be obtained by minimizing the objective function given by

\begin{equation}
\Psi(\mathbf{p}, \mathbf{q}) =  \parallel \mathbf{\Delta T}^o - \mathbf{\Delta T} (\mathbf{p}, \mathbf{q}) \parallel_{2}^{2},
\label{eq:misfit}
\end{equation}
where $\Psi(\mathbf{p}, \mathbf{q})$ is the data misfit, which is the Euclidean norm of the difference between the $\mathbf{\Delta T}^o$ and $\mathbf{\Delta T} (\mathbf{p}, \mathbf{q})$.

The procedure of finding a set of magnetic moment $\mathbf{p}^\sharp$ and magnetization direction $\mathbf{q}^\sharp$ which minimize the equation \ref{eq:misfit} consists to solve an inverse problem for estimating a set of parameters in two steps. Therefore, we split the inverse problem in a mixed solution of two systems of equations. The first one solves a linear system for estimating the part of the magnetic moment. Secondly, the problem is solved through a non-linear process to calculate successive approximations for the part of the magnetization direction at each iteration along the process. 

However, at the $k$th iteration, we impose positivity constraint on the magnetic-moment distribution estimate $\mathbf{p}^k$ by solving the following constrained problem of

\begin{equation}
	\begin{aligned}
		& \text{minimizing}
		& &\lVert \Delta \mathbf{T}^o - \mathbf{G}(\mathbf{q}_{k-1}) \mathbf{p}^k \rVert_{2}^{2} \\
		& \text{subject to}
		& & \mathbf{p}^k \geqslant 0
	\end{aligned}
	\label{eq:positivity}
\end{equation}
where $\mathbf{G}(\mathbf{q}_{k-1})$ is the $N \times M$ matrix defined in equation \ref{eq:tfa_pred},
$\| \cdot \|_{2}^{2}$ represents the squared Euclidean norm and $\mathbf{p}^k \geqslant 0$ means that the magnetic moments of all equivalent sources are positive. This problem is solved by using the nonnegative least squares (NNLS) proposed by (CITAR LAWSON HANSON 1974). In other words, we solve a linear system with positivity constraint at each $k$th iteration given by the equation 

\begin{equation}
	\mathbf{p}^k = \left(\mathbf{G}_{p}^{(k)T} \mathbf{G}_{p}^{(k)}\right)^{-1} \mathbf{G}_{p}^{(k)T}  \Delta \mathbf{T}^o
	\label{eq:linsys_p}
\end{equation}
where $\mathbf{G}_p^{(k)}$ is the magnetic-moment sensitivity matrix at the $k$th iteration. The elements of this matrix are composed by derivative of equation \ref{eq:tfa_pred_pos_i} in relation of $j$th element of the vector $\mathbf{p}^k$.

After estimating the magnetic-moment distribution $\mathbf{p}^k$ at the $k$th iteration using the previous estimate $\mathbf{q}_{k-1}$ for the magnetization direction, we estimate a new vector $\mathbf{q}^{k}$ by solving an unconstrained nonlinear inverse problem of minimizing the squared Euclidean norm of the difference between the observed and predicted total-field anomalies. In this nonlinear inversion we use the Levenberg-Marquardt method \citep{aster_2005}. The iterative process stops when the squared Euclidean norm of the difference between the observed data $\Delta \mathbf{T}^{o}$ and predicted data $\Delta\mathbf{T}(\mathbf{p}, \mathbf{q})$ (equation \ref{eq:tfa_pred}) is invariant along succesive iterations.

%
%However, we have to apply successive corrections $\Delta \mathbf{q}^k$ for calculating the magnetization direction vector $\mathbf{q}^k$ at each $k$th iteration. It is calculated using the Levenberg-Marquardt method (\cite{aster2005}), that is given by equation   
%
%\begin{equation}
%	\Delta \mathbf{q}^k = (\mathbf{G}_{q}^{kT} \mathbf{G}_{q}^k + \lambda \mathbf{I})^{-1} \mathbf{G}_{q}^{kT}  \mathbf{r}^k
%	\label{eq:linsys_q}
%\end{equation}
%where $\mathbf{G}_q^k$ is the sensitivity matrix of the part of magnetization direction of equivalent sources at the $k$th iteration, $\lambda$ is the Marquardt parameter that is update along the iterative process and $\mathbf{I}$ is a indentity matrix. The elements of the matrix $\mathbf{G}_q^k$ are composed by derivative of equation \ref{eq:tfa_pred_pos_i} in relation of each component of the magnetization direction vector $\mathbf{q}^k$, that are the inclination and declination, respectively. The correction is given by
%
%\begin{equation}
%	\mathbf{q}^{k+1} = \mathbf{q}^k + \Delta\mathbf{q}^k,
%	\label{eq:q_correction}
%\end{equation}
%and residual $\mathbf{r}^k$ is equal to 
%
%\begin{equation}
%	\mathbf{r}^k = \Delta \mathbf{T^o} -  \Delta \mathbf{T} (\mathbf{p}^k, \mathbf{q}^k).
%	\label{eq:residual_k}
%\end{equation}
%The iterative process stops when the data-misfit function (equation \ref{eq:misfit}) is invariant along succesive iterations according a stop criterion.

% % Generalization of all-positive equivalent sources 




