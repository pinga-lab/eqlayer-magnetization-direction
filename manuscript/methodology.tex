\section{Methodology}
\label{sec:methodology}

\subsection{Fundamentals of magnetic equivalent layer and the positive magnetic-moment distribution}
\label{subsec:mag_eqlayer}

%%% Continuous layer
Considering a Cartesian coordinate system with $x$-, $y$- and $z$-axis being oriented to north, east and downward, respectively. Let $\Delta T_i \equiv \Delta T (x_i,y_i,z_i)$ be the total field anomaly, at the $i$th position $(x_i,y_i,z_i)$, produced by a continuos layer located below the observation plane on the depth $z_c$, where $z_c > z_i$, and $p(x',y',z_c)$ is the distribution of magnetic dipoles per unit area over the layer surface. In this case, the total-field anomaly produced by a continuous layer is given by equation 

\begin{equation}
\Delta T_i = \int \limits_{-\infty}^{+\infty } \int \limits_{-\infty}^{+\infty }  p(x',y',z_c)  [\gamma_m \hat{\mathbf{F}}_0^T \mathbf{H}(x_i,y_i,z_i,x',y',z_c) \,\hat{\mathbf{h}}(\mathbf{q})] dx' \,dy',
\label{eq:continuous_layer}
\end{equation}
where $\gamma_m$ is a constant proportional to the vacuum permeability, $\hat{\mathbf{F}}_0$ is a unit vector with the same direction of the main geomagnetic field given by

\begin{equation}
	\hat{\mathbf{F}}_0 =
	\left[ \begin{array}{c}
		 \cos I \cos D \\
		 \cos I \sin D \\
		 \sin I     
	\end{array} \right] ,
	\label{eq:main_field}
\end{equation}
where $I$ and $D$ are the inclination and declination, respectively, and $\mathbf{H}(x_i,y_i,z_i,x',y',z_c)$ is a $3 \times 3$ dimensional matrix equal to  
\begin{equation}
\mathbf{H}(x_i,y_i,z_i,x',y',z_c) =
\left[ \begin{array}{ccc}
\partial_{xx} \phi & \partial_{xy} \phi &\partial_{xz} \phi \\  \partial_{yx} \phi & \partial_{yy} \phi &\partial_{yz} \phi \\  \partial_{zx} \phi &\partial_{zy}\phi  & \partial_{zz} \phi    
\end{array} \right] ,
\label{eq:H}
\end{equation}
where $\partial_{\alpha \beta} \phi$, $\alpha = x, y, z$ and $\beta = x, y, z$, is the second derivative of the scalar function 

\begin{equation}
\phi (x_i,y_i,z_i,x',y',z_c) = \frac{1}{[(x_i-x')^2 + (y_i-y')^2 + (z_i-z_c)^2]^{\frac{1}{2}}} .
\label{eq:phi}
\end{equation}
with respect to the Cartesian coordinates $x_i$, $y_i$ and $z_i$ of the observation points. The $\hat{\mathbf{h}}(\mathbf{q})$ is a unit vector with the magnetization direction of the layer given by 

\begin{equation}
	\hat{\mathbf{h}}(\mathbf{q}) =
	\left[ \begin{array}{c}
		\cos \tilde{\i} \cos \tilde{d} \\
		\cos \tilde{\i} \sin \tilde{d}\\
		\sin \tilde{\i}
	\end{array} \right] 
	\label{eq:mag_vec}
\end{equation}
and $\mathbf{q}$ is a $2 \times 1$ vector with components given by 

\begin{equation}
	\mathbf{q} =
	\left[ \begin{array}{c}
		\tilde{\i} \\ 
		\tilde{d} 
	\end{array} \right] ,
	\label{eq:q_vector}
\end{equation}
where $\tilde{\i} $ and $\tilde{d} $ is the inclination and declination of magnetization of the layer, respectively. We can also notice that the vector defined in equation \ref{eq:mag_vec} represents the uniform magnetization direction on the layer. For convenience, this unit vector can be rewritten as follows

\begin{equation}
\hat{\mathbf{h}}(\mathbf{q}) = \mathbf{R}\hat{\mathbf{m}} \: ,
\label{eq:h-rotation-matrix}
\end{equation}
where $\hat{\mathbf{m}}$ defines the uniform magnetization direction of an abitrary magnetic source and $\mathbf{R}$ is a $3 \times 3$ matrix obtained from Euler's rotation theorem. This theorem states that any rotation can be parametrized by using three parameters called Euler angles (CITAR GOLDSTEIN). That is, if the unit vector $\hat{\mathbf{h}}(\mathbf{q})$ (equation \ref{eq:mag_vec}) has the same direction as unit vector  $\hat{\mathbf{m}}$ in the direction of the magnetic source, the matrix $\mathbf{R}$ (equation \ref{eq:h-rotation-matrix}) is equal to identity. For this reason, the total-field anomaly produced by equivalent layer at the $i$th position $(x_i,y_i,z_i)$ (equation \ref{eq:continuous_layer}) can be rewritten as 

\begin{equation}
\Delta T_i = \int \limits_{-\infty}^{+\infty } \int \limits_{-\infty}^{+\infty }  p(x',y',z_c)  [\gamma_m \hat{\mathbf{F}}_0^T \mathbf{H}(x_i,y_i,z_i,x',y',z_c) \,\hat{\mathbf{m}}] dx' \,dy',
\label{eq:continuous_layer_source}
\end{equation} 
which represents the total-field anomaly produced by continuous layer with the same direction of the arbitrary magnetic source.   

%% Explaining the positive property on magnetic layer

%Notice that, if the unit vector $\hat{\vect{h}}$ (equations \ref{eq:h_hat} and \ref{eq:h-hat-rotation-matrix}) has the same direction as the uniform magnetization of the source, the rotation matrix $\vect{R}$ (equation 
%\ref{eq:h-hat-rotation-matrix}) is equal to the identity.
%In this case, by combining equations \ref{eq:generalized-integral},
%\ref{eq:g-mag-dipole} and \ref{eq:h-hat-rotation-matrix}, 
%the total-field anomaly $\Delta T(x_{i}, y_{i}, z_{i})$
%(equation \ref{eq:tfanomaly-general}) can be rewritten as
%\begin{equation}
%\Delta T(x_{i}, y_{i}, z_{i})
%= \int \limits_{-\infty}^{+\infty}
%\int \limits_{-\infty}^{+\infty}
%p(x, y, z_{c}) \,
%\left[ c_{m} \, \frac{\mu_{0}}{4\pi} \,
%\hat{\vect{F}}^{\top} \vect{H} \, 
%\hat{\vect{m}} \right] \,
%dx \, dy \: ,
%\label{eq:tf-p-continuous-mag-positive}
%\end{equation}
%which represents the total-field anomaly produced by
%a continuous layer with the same magnetization direction
%as the true magnetic source. In this case, according to equation 
%\ref{eq:tf-p-continuous-mag-positive}, the RTP anomaly 
%$\Delta T^{P}(x_{i}, y_{i}, z_{i})$ (equation \ref{eq:rtp_anomaly_true}) 
%can be rewritten as follows:
%\begin{equation}
%\Delta T^{P}(x_{i}, y_{i}, z_{i})
%= \int \limits_{-\infty}^{+\infty}
%\int \limits_{-\infty}^{+\infty}
%p(x, y, z_{c}) \,
%\left[ c_{m} \, \frac{\mu_{0}}{4\pi} \,
%\partial_{zz}\frac{1}{r} \, \right] dx \, dy \: ,
%\label{eq:rtp-p-continuous-mag-positive}
%\end{equation}
%where $\partial_{zz}\frac{1}{r}$ is the second derivative of the inverse
%distance function (equation \ref{eq:inv-r}) with respect to $z$.
%By comparing equations \ref{eq:rtp_anomaly_true} and 
%\ref{eq:rtp-p-continuous-mag-positive}, we obtain
%\begin{equation}
%m \: \partial_{zz} \phi(x_{i}, y_{i}, z_{i})
%= \int \limits_{-\infty}^{+\infty}
%\int \limits_{-\infty}^{+\infty}
%p(x, y, z_{c}) \,
%\partial_{zz} \frac{1}{r} \, dx \, dy \: ,
%\label{eq:p-continuous-delzz-theta}
%\end{equation}
%where $m$ is the magnetization intensity of the source, $\partial_{zz} \phi(x_{i}, y_{i}, z_{i})$ is the second derivative, evaluated at the observation point $(x_{i},y_{i},z_{i})$, of the 
%harmonic function $\phi(x, y, z)$ (equation \ref{eq:phi}) 
%with respect to the variable $z$. In this case, the integration in the function $\phi(x, y, z)$ is conducted over the volume of the magnetic source.
%
%Notice that equation \ref{eq:p-continuous-delzz-theta} can be
%obtained by differentiating the following equation
%\begin{equation}
%m \: \partial_{z} \phi(x_{i}, y_{i}, z_{i})
%= \int \limits_{-\infty}^{+\infty}
%\int \limits_{-\infty}^{+\infty}
%\frac{p(x, y, z_{c}) \, (z_{c} - z_{i})}
%{\left[(x - x_{i})^{2} +
%	(y - y_{i})^{2} +
%	(z_{c} - z_{i})^{2} \right]^{\frac{3}{2}}} \, 
%dx \, dy \: , \quad z_c > z_i \: ,
%\label{eq:p-continuous-delz-theta}
%\end{equation}
%with respect to the vertical coordinate $z_{i}$ of the observation point.
%It is worth noting that the physical property $p(x, y, z_{c})$ is a function of the coordinates $(x, y, z_{c})$ on the horizontal plane defining the double layer and, consequently, does not depend on the coordinates $(x_{i}, y_{i}, z_{i})$ of the observation point.
%Then, according to the classical upward continuation integral (equation \ref{eq:upward-continuation-integral}), we conclude that the physical property $p(x, y, z_{c})$ in equation \ref{eq:p-continuous-delz-theta} assumes the particular form
%\begin{equation}
%p(x, y, z_{c}) = 
%\frac{m}{2\pi} \: \partial_{z} \phi(x, y, z_{c}) \: ,
%\label{eq:p-continuous-dzphi}
%\end{equation}
%where $\partial_{z} \phi(x, y, z_{c})$ is the first derivative, evaluated on the equivalent layer, of the function $\phi(x, y, z)$ (equation \ref{eq:phi}) 
%with respect to the variable $z$.
%The most striking aspect of equation \ref{eq:p-continuous-dzphi} is that it represents a particular physical property distribution that is defined as the product of a positive constant $\frac{m}{2\pi}$ and the function $\partial_{z}\phi(x, y, z_{c})$, which is positive at all points $(x, y, z_{c})$ on the planar equivalent layer. Consequently, this particular physical property distribution is positive at all points $(x, y, z_{c})$ on the equivalent layer.
%
%Equation \ref{eq:p-continuous-dzphi} can be rewritten as follows
%\begin{equation}
%p(x, y, z_{c}) = 
%\frac{1}{2\pi} \frac{m}{c_{g} \, \rho \, G} \: \delta g(x, y, z_{c}) \: ,
%\label{eq:p-continuous-mag-positive}
%\end{equation}
%where $\delta g(x, y, z_{c})$ is the gravity disturbance (equation \ref{eq:gravity-disturbance}) 
%that would be produced by the source on the planar surface defining the equivalent layer
%if it have a constant density contrast $\rho$.
%We call attention that equation \ref{eq:p-continuous-mag-positive}
%was deduced by considering that the magnetization intensity $m$ and 
%density $\rho$ are constants throughout the source.
%This relationship is similar to that presented by 
%\citet{pedersen1991} and \citet{li-nabighian-oldenburg2014}.
%By following different approaches in the wavenumber domain, 
%they proved the existence of an all-positive magnetic 
%moment distribution within a planar and continuous layer of dipoles.
%Their approach, however, is valid only for the case in which 
%the observed total-field anomaly is produced by magnetic 
%sources having a purely induced magnetization in the vertical direction.
%They also considered a planar layer which is 
%parallel to a horizontal plane containing the observed total-field anomaly. 
%Under these assumptions, \citet{pedersen1991} and 
%\citet{li-nabighian-oldenburg2014} concluded that the
%magnetic moment distribution within the continuous 
%layer is all-positive and proportional to the 
%pseudogravity anomaly produced by the source on the plane 
%defining the layer.
%Equation \ref{eq:p-continuous-mag-positive} generalizes this 
%positivity condition because it 
%(1) does not impose an induced magnetization within the equivalent layer,
%(2) holds true for all cases in which the magnetization of 
%the planar equivalent layer has the same direction as the uniform 
%magnetization of the source, whenever it is purely induced or not, 
%and (3) does not requireThe unit vector $\hat{\vect{h}}$ (equation \ref{eq:h_hat}) defines the uniform magnetization direction on the layer. For convenience, this unit vector is rewritten as follows
%\begin{equation}
%\hat{\vect{h}} = \vect{R} \hat{\vect{m}} \: ,
%\label{eq:h-hat-rotation-matrix}
%\end{equation}
%where $\hat{\vect{m}}$ (equation \ref{eq:tfanomaly-general})
%defines the uniform magnetization direction of the source and 
%$\vect{R}$ is a $3 \times 3$ matrix obtained from Euler's 
%rotation theorem. This theorem states that any rotation can be 
%parametrized by using three parameters called Euler angles
%\citep{goldstein1980}.
%
%Notice that, if the unit vector $\hat{\vect{h}}$ (equations \ref{eq:h_hat} and \ref{eq:h-hat-rotation-matrix}) has the same direction as the uniform magnetization of the source, the rotation matrix $\vect{R}$ (equation 
%\ref{eq:h-hat-rotation-matrix}) is equal to the identity.
%In this case, by combining equations \ref{eq:generalized-integral},
%\ref{eq:g-mag-dipole} and \ref{eq:h-hat-rotation-matrix}, 
%the total-field anomaly $\Delta T(x_{i}, y_{i}, z_{i})$
%(equation \ref{eq:tfanomaly-general}) can be rewritten as
%\begin{equation}
%\Delta T(x_{i}, y_{i}, z_{i})
%= \int \limits_{-\infty}^{+\infty}
%\int \limits_{-\infty}^{+\infty}
%p(x, y, z_{c}) \,
%\left[ c_{m} \, \frac{\mu_{0}}{4\pi} \,
%\hat{\vect{F}}^{\top} \vect{H} \, 
%\hat{\vect{m}} \right] \,
%dx \, dy \: ,
%\label{eq:tf-p-continuous-mag-positive}
%\end{equation}
%which represents the total-field anomaly produced by
%a continuous layer with the same magnetization direction
%as the true magnetic source. In this case, according to equation 
%\ref{eq:tf-p-continuous-mag-positive}, the RTP anomaly 
%$\Delta T^{P}(x_{i}, y_{i}, z_{i})$ (equation \ref{eq:rtp_anomaly_true}) 
%can be rewritten as follows:
%\begin{equation}
%\Delta T^{P}(x_{i}, y_{i}, z_{i})
%= \int \limits_{-\infty}^{+\infty}
%\int \limits_{-\infty}^{+\infty}
%p(x, y, z_{c}) \,
%\left[ c_{m} \, \frac{\mu_{0}}{4\pi} \,
%\partial_{zz}\frac{1}{r} \, \right] dx \, dy \: ,
%\label{eq:rtp-p-continuous-mag-positive}
%\end{equation}
%where $\partial_{zz}\frac{1}{r}$ is the second derivative of the inverse
%distance function (equation \ref{eq:inv-r}) with respect to $z$.
%By comparing equations \ref{eq:rtp_anomaly_true} and 
%\ref{eq:rtp-p-continuous-mag-positive}, we obtain
%\begin{equation}
%m \: \partial_{zz} \phi(x_{i}, y_{i}, z_{i})
%= \int \limits_{-\infty}^{+\infty}
%\int \limits_{-\infty}^{+\infty}
%p(x, y, z_{c}) \,
%\partial_{zz} \frac{1}{r} \, dx \, dy \: ,
%\label{eq:p-continuous-delzz-theta}
%\end{equation}
%where $m$ is the magnetization intensity of the source, $\partial_{zz} \phi(x_{i}, y_{i}, z_{i})$ is the second derivative, evaluated at the observation point $(x_{i},y_{i},z_{i})$, of the 
%harmonic function $\phi(x, y, z)$ (equation \ref{eq:phi}) 
%with respect to the variable $z$. In this case, the integration in the function $\phi(x, y, z)$ is conducted over the volume of the magnetic source.
%
%Notice that equation \ref{eq:p-continuous-delzz-theta} can be
%obtained by differentiating the following equation
%\begin{equation}
%m \: \partial_{z} \phi(x_{i}, y_{i}, z_{i})
%= \int \limits_{-\infty}^{+\infty}
%\int \limits_{-\infty}^{+\infty}
%\frac{p(x, y, z_{c}) \, (z_{c} - z_{i})}
%{\left[(x - x_{i})^{2} +
%	(y - y_{i})^{2} +
%	(z_{c} - z_{i})^{2} \right]^{\frac{3}{2}}} \, 
%dx \, dy \: , \quad z_c > z_i \: ,
%\label{eq:p-continuous-delz-theta}
%\end{equation}
%with respect to the vertical coordinate $z_{i}$ of the observation point.
%It is worth noting that the physical property $p(x, y, z_{c})$ is a function of the coordinates $(x, y, z_{c})$ on the horizontal plane defining the double layer and, consequently, does not depend on the coordinates $(x_{i}, y_{i}, z_{i})$ of the observation point.
%Then, according to the classical upward continuation integral (equation \ref{eq:upward-continuation-integral}), we conclude that the physical property $p(x, y, z_{c})$ in equation \ref{eq:p-continuous-delz-theta} assumes the particular form
%\begin{equation}
%p(x, y, z_{c}) = 
%\frac{m}{2\pi} \: \partial_{z} \phi(x, y, z_{c}) \: ,
%\label{eq:p-continuous-dzphi}
%\end{equation}
%where $\partial_{z} \phi(x, y, z_{c})$ is the first derivative, evaluated on the equivalent layer, of the function $\phi(x, y, z)$ (equation \ref{eq:phi}) 
%with respect to the variable $z$.
%The most striking aspect of equation \ref{eq:p-continuous-dzphi} is that it represents a particular physical property distribution that is defined as the product of a positive constant $\frac{m}{2\pi}$ and the function $\partial_{z}\phi(x, y, z_{c})$, which is positive at all points $(x, y, z_{c})$ on the planar equivalent layer. Consequently, this particular physical property distribution is positive at all points $(x, y, z_{c})$ on the equivalent layer.
%
%Equation \ref{eq:p-continuous-dzphi} can be rewritten as follows
%\begin{equation}
%p(x, y, z_{c}) = 
%\frac{1}{2\pi} \frac{m}{c_{g} \, \rho \, G} \: \delta g(x, y, z_{c}) \: ,
%\label{eq:p-continuous-mag-positive}
%\end{equation}
%where $\delta g(x, y, z_{c})$ is the gravity disturbance (equation \ref{eq:gravity-disturbance}) 
%that would be produced by the source on the planar surface defining the equivalent layer
%if it have a constant density contrast $\rho$.
%We call attention that equation \ref{eq:p-continuous-mag-positive}
%was deduced by considering that the magnetization intensity $m$ and 
%density $\rho$ are constants throughout the source.
%This relationship is similar to that presented by 
%\citet{pedersen1991} and \citet{li-nabighian-oldenburg2014}.
%By following different approaches in the wavenumber domain, 
%they proved the existence of an all-positive magnetic 
%moment distribution within a planar and continuous layer of dipoles.
%Their approach, however, is valid only for the case in which 
%the observed total-field anomaly is produced by magnetic 
%sources having a purely induced magnetization in the vertical direction.
%They also considered a planar layer which is 
%parallel to a horizontal plane containing the observed total-field anomaly. 
%Under these assumptions, \citet{pedersen1991} and 
%\citet{li-nabighian-oldenburg2014} concluded that the
%magnetic moment distribution within the continuous 
%layer is all-positive and proportional to the 
%pseudogravity anomaly produced by the source on the plane 
%defining the layer.
%Equation \ref{eq:p-continuous-mag-positive} generalizes this 
%positivity condition because it 
%(1) does not impose an induced magnetization within the equivalent layer,
%(2) holds true for all cases in which the magnetization of 
%the planar equivalent layer has the same direction as the uniform 
%magnetization of the  that the observed total-field anomaly data be
%on a plane.

















\subsection{Forward problem for magnetic equivalent-layer technique}

%% Discretizing layer
However, in practical situations, its not possible to determine a continuous magnetic-moment distribution $p(x',y',z_c)$ over the layer as shown in equation \ref{eq:continuous_layer}. For this reason, the continuous equivalent layer have to be approximated  by a discrete set of $M$ dipoles with unit volume located at a constant depth $z = z_c$. Let $\mathbf{p}$ be an $M$-dimensional parameter vector, whose $j$th element $p_j$ is the magnetic intensity of the $j$th dipole and $\mathbf{q}$ be a vector containing the inclination $\tilde{i}$ and declination $\tilde{d}$ of all dipole, analogously to equation \ref{eq:q_vector}. Mathematically, by discretizing the integrand of equation \ref{eq:continuous_layer}, the total-field anomaly produced by equivalent layer at the point $(x_i,y_i,z_i)$ is given by     

\begin{equation}
\Delta T_i (\mathbf{p},\mathbf{q}) = \sum_{j=1}^{M} p_j g_{ij} (\mathbf{q})
\label{eq:tfa_pred_pos_i}
\end{equation}    
where 

\begin{equation}
g_{ij} (\mathbf{q})  = \gamma_m \hat{\mathbf{F}}_0^T \mathbf{H}_{ij} \hat{\mathbf{h}}(\mathbf{q})
\label{eq:g_ij}
\end{equation}
is an harmonic function representing the total-field anomaly produced at the $i$th position $(x_i,y_i,z_i)$ by a dipole located at $(x_j,y_j,z_c)$ with unitary magnetic-moment intensity. The matrix $\mathbf{H}_{ij}$ is formed by the second derivatives of a function $\phi_{ij}$ that depends on the inverse of the scalar function $r_{ij} = [(x_i-x_j)^2 + (y_i-y_j)^2 + (z_i-z_c)^2]^{1/2}$, analogously to equation \ref{eq:H} and \ref{eq:phi}. In matrix notation, the equation \ref{eq:tfa_pred_pos_i} can be represented as 

\begin{equation}
 \mathbf{\Delta T} (\mathbf{p}, \mathbf{q}) = \mathbf{G}(\mathbf{q}) \mathbf{p}
\label{eq:tfa_pred}
\end{equation}
where $\mathbf{G}(\mathbf{q})$ is an $N \times M$ matrix whose $ij$th element is defined by the harmonic function $g_{ij}(\mathbb{q})$ (equation \ref{eq:g_ij}) and $\mathbf{\Delta T} (\mathbf{p}, \mathbf{q})$ is an $N \times 1$ vector whose the $i$th element is the predicted total-field anomaly $\Delta T_i (\mathbf{p},\mathbf{q})$ (equation \ref{eq:tfa_pred_pos_i}). As can be noticed from equation \ref{eq:tfa_pred_pos_i}-\ref{eq:tfa_pred}, the predicted total-field anomaly produced by equivalent layer has a linear relation with the magnetic moment $\mathbf{p}$ and a nonlinear relation with the magnetization direction $\mathbf{q}$. 

\subsection{Iterative process for magnetization estimation}

%%% Defining the objective function

Let $\mathbf{\Delta T}^o$ be an \textit{N}-dimensional vector whose $i$th element $\Delta T_i^o$ is the total field anomaly observation produced by a magnetic source at the point $(x_i,y_i,z_i)$, $i = 1, \ldots, N$. The estimation of the magnetic moments $\mathbf{p}$ and the magnetization direction $\mathbf{q}$ consists to formulate an inverse problem by imposing a positivity constraint on the magnetic-moment distribution. It can be performed by minimizing the difference between the observed data $\mathbf{\Delta T}^o$ and the predicted data $\Delta T (\mathbf{p}, \mathbf{q})$ (equation \ref{eq:tfa_pred}) by imposing a positivity constraint. 

In other words, a stable estimates $\mathbf{p}^\sharp$ and $\mathbf{q}^\sharp$ can be obtained by minimizing the objective function given by

\begin{equation}
\Psi(\mathbf{p}, \mathbf{q}) =  \parallel \mathbf{\Delta T}^o - \mathbf{\Delta T} (\mathbf{p}, \mathbf{q}) \parallel_{2}^{2},
\label{eq:misfit}
\end{equation}
where $\Psi(\mathbf{p}, \mathbf{q})$ is the data misfit, which is the Euclidean norm of the difference between the $\mathbf{\Delta T}^o$ and $\mathbf{\Delta T} (\mathbf{p}, \mathbf{q})$.

%%% Explaining the iterative process

The procedure of finding a set of magnetic moment $\mathbf{p}^\sharp$ and magnetization direction $\mathbf{q}^\sharp$ which minimize the equation \ref{eq:misfit} consists to solve an inverse problem for estimating a set of parameters in two steps. Therefore, we split the inverse problem in a mixed solution of two systems of equations. The first one solves a linear system for estimating the part of the magnetic moment. Secondly, the problem is solved through a non-linear process to calculate successive approximations for the part of the magnetization direction at each iteration along the process. 

However, at the $k$th iteration, we impose positivity constraint on the magnetic-moment distribution estimate $\mathbf{p}^k$ by solving the following constrained problem of

\begin{equation}
	\begin{aligned}
		& \text{minimizing}
		& &\lVert \Delta \mathbf{T}^o - \mathbf{G}(\mathbf{q}_{k-1}) \mathbf{p}^k \rVert_{2}^{2} \\
		& \text{subject to}
		& & \mathbf{p}^k \geqslant 0
	\end{aligned}
	\label{eq:positivity}
\end{equation}
where $\mathbf{G}(\mathbf{q}_{k-1})$ is the $N \times M$ matrix defined in equation \ref{eq:tfa_pred},
$\| \cdot \|_{2}^{2}$ represents the squared Euclidean norm and $\mathbf{p}^k \geqslant 0$ means that the magnetic moments of all equivalent sources are positive. This problem is solved by using the nonnegative least squares (NNLS) proposed by (CITAR LAWSON HANSON 1974). In other words, we solve a linear system with positivity constraint at each $k$th iteration given by the equation 

\begin{equation}
	\mathbf{p}^k = \left(\mathbf{G}_{p}^{(k)T} \mathbf{G}_{p}^{(k)}\right)^{-1} \mathbf{G}_{p}^{(k)T}  \Delta \mathbf{T}^o
	\label{eq:linsys_p}
\end{equation}
where $\mathbf{G}_p^{(k)}$ is the magnetic-moment sensitivity matrix at the $k$th iteration. The elements of this matrix are composed by derivative of equation \ref{eq:tfa_pred_pos_i} in relation of $j$th element of the vector $\mathbf{p}^k$.

After estimating the magnetic-moment distribution $\mathbf{p}^k$ at the $k$th iteration using the previous estimate $\mathbf{q}_{k-1}$ for the magnetization direction, we estimate a new vector $\mathbf{q}^{k}$ by solving an unconstrained nonlinear inverse problem of minimizing the squared Euclidean norm of the difference between the observed and predicted total-field anomalies. In this nonlinear inversion we use the Levenberg-Marquardt method (CITAR ASTER). That is, we calculate at each $k$th iteration the step $\Delta \mathbf{q}^k$ for the magnetization direction by using the equation

\begin{equation}
	\Delta \mathbf{q}^k = (\mathbf{G}_{q}^{(k)T} \mathbf{G}_{q}^{(k)} + \lambda \mathbf{I})^{-1} \mathbf{G}_{q}^{(k)T}  \mathbf{r}^k
	\label{eq:linsys_q}
\end{equation}
where $\lambda$ is the Marquardt parameter that is updated along the iterative process, $\mathbf{I}$ is a indentity matrix, and  the residual at the $k$th iteration $r^k = \Delta \mathbf{T}^o - \mathbf{\Delta T} (\mathbf{p}^k, \mathbf{q}^{k-1})$. $\mathbf{G}_q^k$ is a sensitivity matrix of the magnetization direction part, whose elements are composed by derivative of equation \ref{eq:tfa_pred_pos_i} in relation of each component of the vector $\mathbf{q}^k$, that are the inclination and declination, respectively. The iterative process stops when the squared Euclidean norm of the difference between the observed data $\Delta \mathbf{T}^{o}$ and predicted data $\Delta\mathbf{T}(\mathbf{p}, \mathbf{q})$ (equation \ref{eq:tfa_pred}) is invariant along succesive iterations (Figure \ref{fig:scheme_LM_NNLS}).


%However, we have to apply successive corrections $\Delta \mathbf{q}^k$ for calculating the magnetization direction vector $\mathbf{q}^k$ at each $k$th iteration. It is calculated using the Levenberg-Marquardt method (\cite{aster2005}), that is given by equation   
%
%\begin{equation}
%	\Delta \mathbf{q}^k = (\mathbf{G}_{q}^{kT} \mathbf{G}_{q}^k + \lambda \mathbf{I})^{-1} \mathbf{G}_{q}^{kT}  \mathbf{r}^k
%	\label{eq:linsys_q}
%\end{equation}
%where $\mathbf{G}_q^k$ is the sensitivity matrix of the part of magnetization direction of equivalent sources at the $k$th iteration, $\lambda$ is the Marquardt parameter that is update along the iterative process and $\mathbf{I}$ is a indentity matrix. The elements of the matrix $\mathbf{G}_q^k$ are composed by derivative of equation \ref{eq:tfa_pred_pos_i} in relation of each component of the magnetization direction vector $\mathbf{q}^k$, that are the inclination and declination, respectively. The correction is given by
%
%\begin{equation}
%	\mathbf{q}^{k+1} = \mathbf{q}^k + \Delta\mathbf{q}^k,
%	\label{eq:q_correction}
%\end{equation}
%and residual $\mathbf{r}^k$ is equal to 
%
%\begin{equation}
%	\mathbf{r}^k = \Delta \mathbf{T^o} -  \Delta \mathbf{T} (\mathbf{p}^k, \mathbf{q}^k).
%	\label{eq:residual_k}
%\end{equation}
%The iterative process stops when the data-misfit function (equation \ref{eq:misfit}) is invariant along succesive iterations according a stop criterion.
