\section{Methodology}
\label{sec:methodology}

% % Explaining the continuous magnetic equivalent layer  

Considering a Cartesian coordinate system with $x$-, $y$- and $z$-axis being oriented to north, east and downward, respectively. Let $\Delta T_i \equiv \Delta T (x_i,y_i,z_i)$ be the total field anomaly, at the $i$-th position $(x_i,y_i,z_i)$, produced by a continuos layer located below the observation plane on the depth $z_c$, where $z_c > z_i$, and $p(x',y',z_c)$ is the distribution of magnetic dipoles per unit area over the layer surface. In this case, the total field anomaly produced by this continuous layer is given by the equation 

\begin{equation}
\Delta T_i = \int \limits_{-\infty}^{+\infty } \int \limits_{-\infty}^{+\infty }  p(x',y',z_c)  [\gamma_m \hat{\mathbf{F}}_0^T \mathbf{H} \,\hat{\mathbf{h}}(\mathbf{q})] dx' \,dy',
\label{eq:continuous_layer}
\end{equation}
where $\gamma_m$ is a constant proportional to the vaccum permeability, $\hat{\mathbf{F}}_0$ is a unit vector with the same direction of the geomagnetic field $\mathbf{F}_0$ and $\mathbf{H}$ is a $3 \times 3$ matrix equal to  

 \begin{equation}
   \mathbf{H} =
   \left[ \begin{array}{ccc}
   \partial_{xx} \phi & \partial_{xy} \phi &\partial_{xz} \phi \\  \partial_{yx} \phi & \partial_{yy} \phi &\partial_{yz} \phi \\  \partial_{zx} \phi &\partial_{zy}\phi  & \partial_{zz} \phi    
   \end{array} \right] ,
   \label{eq:H}
 \end{equation}
where $\partial_{\alpha \beta}\phi$, $\alpha = x, y, z$, $\beta = x, y, z$, is the second derivative of the function 

\begin{equation}
   \phi (x-x', y-y', z-z_c) = \frac{1}{r} ,
   \label{eq:phi}
 \end{equation}
where $r = [(x-x')^2 + (y-y')^2 + (z-z_c)^2]^{1/2}$ and $\hat{\mathbf{h}}(\mathbf{q})$ is a unit vector with the
magnetization direction of the layer that depends on the vector $\mathbf{q}$ given by 

 \begin{equation}
   \mathbf{q} =
   \left[ \begin{array}{c}
   i  \\ 
   d     
   \end{array} \right] ,
   \label{eq:q_vector}
 \end{equation}
 where $i$ and $d$ is the inclination and declination, respectively.

% % Formulating foward problem 


According the equivalent layer technique theory, we can produce the a set of $N$ total field anomaly 
observations generated by a 3D magnetic source by a layer. 


By discretizing 

%It is worth pointing out that the equivalent source is represented by a dipole with unit volume

%$\mathbf{p}$ is the \textit{M}-dimensional vector defined as parameter vector whose $j$th element is the magnetic intensity of the $j$th equivalent source

%$\mathbf{q}$ is a $2 \times 1 $ vector containing the inclination and the declination of the equivalent dipoles 
  

%We consider that the $M$ equivalent sources are distributed with a constant depth $h$ below the observation plane, forming an equivalent layer 

%Mathematically, the effect produced by the equivalent layer at $N$ observation points can be given as  

% % Formulating the inverse problem explaining the iterative method

%Let $\mathbf{\Delta T}^o$ be an \textit{N}-dimensional vector whose $i$th element is the total field anomaly observation produced by a magnetic source at the point $(x_i,y_i,z_i)$, $i = 1, \ldots, N$. 

% % Constraining the magnetic moment of equivalent sources

% % Practical Procedures 

% % Generalization of all-positive equivalent sources 




