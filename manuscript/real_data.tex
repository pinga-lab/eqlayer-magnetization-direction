\section{Application to field data}
\label{sec:real_application}

The Goias alkaline province (GAP) is a region in the central part of Brazil where there are occurences of mafic-ultramafic alkaline magmatism. This region presents a variety of rocks with an extense petrographic types. Throughout the area there are mafic-ultramafic complexes (plutonic intrusions), subvolcaninc alkaline intrusions (diatremes) and volcanic products (kamafugite lava flows) with several dikes. Some of the main alkaline complexes of GAP are: the Montes Claros de Goi\'as, Diorama, C\'orrego dos Bois, Morro do Macaco and Fazenda Buriti. These alkaline intrusions are sorrounded by a Precambrian basement and the Phanerozoic sedimentary rocks of the Paran\'a Basin. \citep{junqueira_brod_2005,carlson_etal_2007,marangoni_mantovani_2013,dutra_etal_2014}. Recent studies indicate the existence of a remarkable remanent magnetization component within these intrusions \citep{marangoni_mantovani_2013,oliveirajr_etal_2015,marangoni_etal_2016,zhang_etal_2018}. 

The aeromagnetic survey has a flight pattern with north-south flight lines spaced from $\sim 500$ m and data acquired at intervals of $ \sim 8$ m along each line, and a constant height of $100$ m from the terrain. The geomagnetic field direction for this area was $-19.5^\circ$ and $-18.5^\circ$ for inclination and declination, respectively. We invert the total-field anomaly (Figure \ref{fig:mc_data_application}a) from the alkaline complex of Montes Claros. To speed up data processing and inversion, we downsampled the data along the flight lines, resulting a grid of $55 \times 32$ points (a total of $N=1787$ observations). This new set up results an approximately $320$ m and $470$ m grid spacing along the x- and y-axes, respectively. We use an equivalent layer composed by a grid of $55 \times 32$ dipoles (a total of $M=1787$ equivalent sources) positioned at a depth of $840$ m below the observation plane ($\sim$ twice the greater grid spacing). The algorithm starts with an initial guess of $-70^\circ$ and $50^\circ$ for the inclination and declination, respectively. Figure \ref{fig:mc_data_application}b shows the predicted data produced by equivalent layer. Figure \ref{fig:mc_data_application}c shows the residuals defined as the difference between the observed data (Figure \ref{fig:mc_data_application}a) and the predicted data (Figure \ref{fig:mc_data_application}b). Note that two small places in Figure \ref{fig:mc_data_application}c where large residuals are clearly apparent may indicate the existence of shallow-seated geological sources with different magnetization direction. However, the histogram of residuals (Figure \ref{fig:mc_data_application}d) is acceptable with its mean of $-14.52$ nT ($\sim 0.1\% $ of the maximum value of total-field anomaly data) and standard deviation of $312.28$ nT ($\sim 2 \% $ of the maximum value of total-field anomaly data). The estimated magnetization direction $\bar{\mathbf{q}}$ has inclination $-50.2^\circ$ and declination $34.9^\circ$. Figure \ref{fig:mc_data_application}e and \ref{fig:mc_data_application}f shows the estimated magnetic-moment distribution $\bar{\mathbf{p}}$ and the convergence of the algorithm. We check the quality of the estimated magnetization direction by computing the reduction-to-pole of the observed total-field anomaly. We can note that the reduced-to-the-pole (RTP) anomaly (Figure \ref{fig:rtp_mc_data}) exhibits predominantly positive values and decays to zero towards the borders of study area. For this reason, we consider that the estimated magnetization direction led to a satisfactory RTP anomaly. We conclude with these results that the all-positive magnetic moment distribution and the estimated magnetization direction produce an acceptable data fit. 
According to \citet{marangoni_mantovani_2013}, laboratory measurements made with 
rock samples of the GAP indicate an average total magnetization direction 
with inclination and declination equal to $-39.0^\circ$ and $1.0^\circ$, 
respectively.
However, \citet{zhang_etal_2018} use aeromagnetic data to estimate a total magnetization direction 
with inclination $-49.0^\circ$ and declination 
$46.0^\circ$ for the same complex of Montes Claros de Goi\'as. 
These differences may be due to the fact that the results obtained with 
rock samples represent the total-magnetization direction of local shallow sources,
whereas the result obtained with airborne data reflect the predominant influence of 
deeper sources at the study area. 
We note that our results, also obtained with airborne data, are very close to 
those obtained by \citet{zhang_etal_2018} for the same complex. Moreover, 
they also confirm the existence of remarkable remanent magnetization for this area. 


