\section{Application to synthetic data}
\label{sec:synt_tests}

We test the proposed method by applying it to three different synthetic data sets simulating complex geological scenarios. The first one is generated by a model containing a set of sources with different geometries, all of them with the same magnetization direction. The second data set is generated by a set of magnetized bodies, but one them is a shallow interfering source with the same magnetization direction. In the third test, we violate the unidirectional approach by simulating an interfering shallow source with different magnetization direction from the others bodies.

For all tests, the simulated data were computed on an irregular grid of $49 \times 25$ points (a total of $N = 1225$ observations) at a constant height of $100 \, m$.  We assume an observation area of extending $12 \, km$ along the x- and y-axis, resulting a grid spacing of $250 \, m$ and $500 \, m$ on x- and y-axis, respectively. The data were contamined with pseudorandom Gaussian noise with zero mean and $10 \, nT$ standard deviation. The main field direction simulated was $I= -40^\circ$ and $D= -22^\circ$ for the inclination and declination, respectively. For the inversion, we use an equivalent layer composed by a grid of $49 \times 25$ dipoles (a total of $M = 1225$ equivalent sources) positioned at a depth of $z_c = 1150 \, m$ below the observation plane ($2.5$ times the greater grid spacing). We use the L-curve to choose the regularizing parameter ($\mu$). The algorithm \ref{cd: LM_NNLS} starts with an initial guess $\mathbf{q}_0 = (-10^\circ,-10^\circ)$ for inclination and declination, respectively.

\subsection{Unidirectional magnetization direction sources}
 
In order to test the methodology, we generate a 3D prism with polygonal cross-section whose the top is positioned at a depth of $450 \, m$ and the bottom $3150 \, m$ with magnetization intensity of $4 \, A/m$. We also generate two spheres with radius equal to $500 \, m$, one of them with the coordinates of center $x_c = 1800 \, m $, $y_c = -1800 \, m $ and $z_c = 1000 \, m$ and the other with $x_c = 800 \, m$, $y_c = 800 \, m$ and $z_c= 1000 \, m$. The magnetization intensity for the two both spheres is equal to  $3 \, A/m$. We produce two rectangular prisms with the same $2.5 \, A/m$ of magnetization intensity. The smaller prism has the top at a depth of $450 \, m$ and side lengths of $1000 \, m$, $700 \, m$ and $500 \, m$ along x-,y- and z-axis, respectively. The greater prism has the top at a depth of $500 \, m$ and side lengths of $1000 \, m$, $2000 \, m$ and $1550 \, m$ along x-,y- and z-axis. All simulated sources have inclination $-25^\circ$ and declination $30^\circ$. The noise-corrupted data is shown in figure \ref{fig:unidir_test}a. 

Figure \ref{fig:unidir_test}b shows the predicted data produced by equivalent layer. 
Figure \ref{fig:unidir_test}c shows the residuals defined as the difference between the simulated data (figure \ref{fig:unidir_test}a) and the predicted data (figure \ref{fig:unidir_test}b). The residuals appear normally distributed with a mean of $-0.30 \, nT$ and a standard deviation of $9.67 \, nT$ as shown in figure \ref{fig:unidir_test}d. The estimated magnetization direction $\mathbf{q}^\sharp$ has inclination $-28.6^\circ$ and declination $30.8^\circ$. Figure \ref{fig:unidir_test}e shows the estimated magnetic-moment distribution $\mathbf{p}^\sharp$. The convergence of the algorithm \ref{cd: LM_NNLS} is shown in figure \ref{fig:unidir_test}f. These results show that the all-positive magnetic-moment distribution and the estimated magnetization direction produce an acceptable data fitting.

\subsection{Unidirectional model with shallow interfering source}

In this section we test the methodology perfomance when exist a shallow interfering source. The model is almost the same as the previous section except the smaller prism. For this purpose, we put the top of the smaller prism at a depth of $150 \, m$ and side lengths of $1000 \, m$, $700 \, m$ and $500 \, m$ along x-,y- and z-axis, respectively. The magnetization intensity for this prism is equal to $1.5 \, A/m$. The magnetization direction of all sources is $-25^\circ$ inclination and declination $30^\circ$, respectively. The synthetic data is shown in figure \ref{fig:unidir_shallow_test}a.

Figure \ref{fig:unidir_shallow_test}b shows the predicted total-field anomaly produced by equivalent layer. Figure \ref{fig:unidir_shallow_test}c shows the residuals defined as the difference between the simulated data (figure \ref{fig:unidir_shallow_test}a) and the predicted data (figure \ref{fig:unidir_shallow_test}b). The residuals appear normally distributed with a mean of $-0.42 \, nT$ and a standard deviation of $10.67 \, nT$ as shown in figure \ref{fig:unidir_shallow_test}d. The estimated magnetization direction $\mathbf{q}^\sharp$ has inclination $-28.7^\circ$ and declination $31.7^\circ$. Figure \ref{fig:unidir_shallow_test}e shows the estimated magnetic-moment distribution $\mathbf{p}^\sharp$. The convergence of the algorithm \ref{cd: LM_NNLS} is shown in figure \ref{fig:unidir_shallow_test}f. Despite the residual located above the shallow magnetic source, we consider the methodology produced a reliable result. Therefore, that the all-positive magnetic-moment distribution and the estimated magnetization direction produce an acceptable data fitting. 

\subsection{Shallow source with different magnetization direction}

In this test we simulate the presence of a shallow interfering body with different-magnetization direction from the others magnetic sources. The shallow prism has the dimension and the intensity magnetization equal to the previous one. The magnetization direction of this prism is $20^\circ$ inclination and declination $-30^\circ$. The others sources of the model has $-25^\circ$ inclination and declination $30^\circ$. The noise-corrupted data is shown in figure \ref{fig:unidir_shallow_diff_test}a.

%Figure \ref{fig:unidir_shallow_test}b shows the predicted total-field anomaly produced by equivalent layer. Figure \ref{fig:unidir_shallow_test}c shows the residuals defined as the difference between the simulated data (figure \ref{fig:unidir_shallow_test}a) and the predicted data (figure \ref{fig:unidir_shallow_test}b). The residuals appear normally distributed with a mean of $-0.42 \, nT$ and a standard deviation of $10.67 \, nT$ as shown in figure \ref{fig:unidir_shallow_test}d. The estimated magnetization direction $\mathbf{q}^\sharp$ has inclination $-28.7^\circ$ and declination $31.7^\circ$. Figure \ref{fig:unidir_shallow_test}e shows the estimated magnetic-moment distribution $\mathbf{p}^\sharp$. The convergence of the algorithm \ref{cd: LM_NNLS} is shown in figure \ref{fig:unidir_shallow_test}f. Despite the residual located above the shallow magnetic source, we consider the methodology produced a reliable result. Therefore, that the all-positive magnetic-moment distribution and the estimated magnetization direction produce an acceptable data fitting.




























