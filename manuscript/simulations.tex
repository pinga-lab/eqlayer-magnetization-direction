\section{Application to synthetic data}
\label{sec:synt_tests}

We test the proposed method by applying it to three different synthetic data sets simulating complex geological scenarios. The first one is generated by a model containing a set of sources with different geometries, all of them with the same magnetization direction. The second data set is generated by a model with different geometries and same magnetization direction, but containing a shallow high-magnetized interfering source. In the third test, we test a violating condition of the unidirectional approach by simulating a model with a set of bodies containing a shallow source with a magnetization direction different from the others.

For all tests, the simulated data were computed on an irregular grid of $50 \times 25$ points (a total of $N = 1250$ observations) at a constant height of $100 \, m$.  We assume an observation area of extending $12 \, km$ along the x- and y-axis, resulting a grid spacing of approximately $250 \, m$ and $500 \, m$ on x- and y-axis, respectively. The data were contamined with pseudorandom Gaussian noise with zero mean and $15 \, nT$ standard deviation. For the inversion, we use an equivalent layer composed by a grid of $50 \times 25$ dipoles (a total of $M = 1250$ equivalent sources) positioned at a depth of $z_c = 1150 \, m$ below the observation plane ($2.5$ times the greater grid spacing).  
 



