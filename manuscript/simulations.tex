\section{Application to synthetic data}
\label{sec:synt_tests}

We test the proposed method by applying it to three different synthetic data sets simulating complex geological scenarios. The first one is generated by a model containing a set of sources with different geometries, all of them with the same magnetization direction. The second data set is generated by a set of magnetized bodies, but one them is a shallow interfering source with the same magnetization direction. In the third test, we violate the unidirectional approach by simulating an interfering shallow source with different magnetization direction from the others bodies.

For all tests, the simulated data were computed on an irregular grid of $50 \times 25$ points (a total of $N = 1250$ observations) at a constant height of $100 \, m$.  We assume an observation area of extending $12 \, km$ along the x- and y-axis, resulting a grid spacing of approximately $250 \, m$ and $500 \, m$ on x- and y-axis, respectively. The data were contamined with pseudorandom Gaussian noise with zero mean and $15 \, nT$ standard deviation. The main field direction simulated was $I= -20^\circ$ and $D= 20^\circ$ for the inclination and declination, respectively. For the inversion, we use an equivalent layer composed by a grid of $50 \times 25$ dipoles (a total of $M = 1250$ equivalent sources) positioned at a depth of $z_c = 1150 \, m$ below the observation plane ($2.5$ times the greater grid spacing). We use the L-curve to choose the regularizing parameter ($\mu$). The algorithm \ref{cd: LM_NNLS} starts with an initial guess $\mathbf{q}_0 = (-10^\circ,-10^\circ)$ for inclination and declination, respectively.

\subsection{Unidirectional magnetization direction sources}
 



